\overfullrule=0pt
%CHAPTER NUMBERING
%1) default = No specification needed
        %Chapters with numbers (1.0, 2.0, etc.) 
        %sections with numbers and sub-numbers (1.1, 1.2, 2.1, 2.2, etc.) 
        %subsections with numbers and sub-numbers (an additional sub-number) (1.1.1, 1.1.2, 2.1.1, 2.1.2, etc)
        %subsubsections with numbers and sub-numbers (two additional sub-numbers) (1.1.1.1, 1.1.1.2, 2.1.1.1, 2.1.1.2, etc.)
%2)'sectionletters'= Changes numbering format
        %Chapters with Roman numerals (I, II, etc.) 
        %sections with letters (A, B) 
        %subsections with numbers (1, 2)
        %subsubsections with lowercase letters (a, b)

%SPECIAL OPTIONS
%VERSION
%1)'final' = Changes all "format warnings" into errors (Final version of the document       

%%%%%%%%%%%%%%%%%%%%%%%%%%%%%%%%%%%%%%%%%%%%%%%%%%%%%%%%%%%%%%%%%%%%%%%%%%%%%%%%%%%%%%%%%%%%%%%%%%%
% Citation Management
%%%%%%%%%%%%%%%%%%%%%%%%%%%%%%%%%%%%%%%%%%%%%%%%%%%%%%%%%%%%%%%%%%%%%%%%%%%%%%%%%%%%%%%%%%%%%%%%%%%
\usepackage[%
    backend      = biber,%
    style        = chem-acs,%
    autocite     = superscript,%
    backref      = true,%
    doi          = true,%
    articletitle = true,%
    chaptertitle = true,%
    biblabel     = brackets,%
    minnames     = 1,%
    maxnames     = 999,%,
    texencoding=ascii
]{biblatex}
\let\cite\autocite
\DefineBibliographyStrings{english}{%
  backrefpage = {page},% originally "cit. on p."
  backrefpages = {pages},%
}
\defbibenvironment{bibliography}
  {\list
     {\printtext[labelnumberwidth]{%
        \printfield{labelprefix}%
        \printfield{labelnumber}}}
     {%
      \setlength{\labelwidth}{\labelnumberwidth}%
      \setlength{\leftmargin}{\labelwidth}%
      \setlength{\labelsep}{\biblabelsep}%
      \addtolength{\leftmargin}{\labelsep}%
      \setlength{\itemsep}{\bibitemsep}%
      \setlength{\parsep}{\bibparsep}}%
      \renewcommand*{\makelabel}[1]{##1}}
  {\endlist}
{\item}
\usepackage[T1]{fontenc}
\addbibresource{references.bib}
%%%%%%%%%%%%%%%%%%%%%%%%%%%%%%%%%%%%%%%%%%%%%%%%%%%%%%%%%%%%%%%%%%%%%%%%%%%%%%%%%%%%%%%%%%%%%%%%%%%
% Default includes
%%%%%%%%%%%%%%%%%%%%%%%%%%%%%%%%%%%%%%%%%%%%%%%%%%%%%%%%%%%%%%%%%%%%%%%%%%%%%%%%%%%%%%%%%%%%%%%%%%%
\usepackage[utf8]{inputenc}
\usepackage{graphicx}
\usepackage{indentfirst}
\usewithpatch{amsmath,amsthm}
% Patches for the packages 'amsmath' and 'amsthm'
\patch{amsmath}
\patch{amsthm}

\usepackage{pdflscape}
\hypersetup{hidelinks}

\makeatletter
\renewcommand{\@biblabel}[1]{[#1]\hfill}
\makeatother
\makeatletter
\def\@hangfrom#1{\setbox\@tempboxa\hbox{{#1}}%
      \hangindent 0pt%\wd\@tempboxa
      \noindent\box\@tempboxa}
\makeatother

%%%%%%%%%%%%%%%%%%%%%%%%%%%%%%%%%%%%%%%%%%%%%%%%%%%%%%%%%%%%%%%%%%%%%%%%%%%%%%%%%%%%%%%%%%%%%%%%%%%
% Additional includes
%%%%%%%%%%%%%%%%%%%%%%%%%%%%%%%%%%%%%%%%%%%%%%%%%%%%%%%%%%%%%%%%%%%%%%%%%%%%%%%%%%%%%%%%%%%%%%%%%%%
\usepackage[acronym]{glossaries-extra}
\setabbreviationstyle[acronym]{long-short}
\usepackage{xcolor}
\definecolor{shiv_purple}{rgb}{0.6       ,  0.19607843,  0.8}
\definecolor{shiv_blue}{rgb}{0.11764706,  0.56470588,  1.}
\definecolor{shiv_green}{rgb}{0.        ,  0.57647059,  0.23529412}
\definecolor{shiv_yellow}{rgb}{0.97647059,  0.75686275,  0.1372549}
\definecolor{shiv_orange}{rgb}{1.        ,  0.54901961,  0.}
\definecolor{shiv_red}{rgb}{0.93333333,  0.20784314,  0.18039216}
\definecolor{shiv_gray}{rgb}{0.72156863,  0.71764706,  0.73333333}
\newcommand{\shiv}[1]{\textcolor{shiv_purple}{\bf [SU #1 ]}}
\newcommand{\citehere}{\textcolor{red}{\bf\textsuperscript{\textdagger}}}
\newcommand{\citeherecomment}[1]{\textcolor{red}{\bf\textsuperscript{\textdagger} #1}}
\newcommand{\todo}[1]{\textcolor{shiv_red}{\bf [TO DO: #1 ]}}

\usepackage{etoolbox}

% Chapter 1
\usepackage{amsfonts}
\usepackage{braket}
% Chapter 2
\usepackage{multirow}
% Chapter 3
%\usepackage{physics}
\usepackage[version=4]{mhchem}
\usepackage{siunitx}
\newcommand{\angstrom}{\mbox{\normalfont\AA}}
%\DeclareSIUnit[]{\angstrom}{\mbox{\text {Å}}}
%\sisetup{load-configurations = abbreviations}
\usepackage{tablefootnote}
\usepackage{booktabs}
\usepackage{threeparttable}
% Chapter 5
\newcommand{\shivten}[5]{% #1 presuperscript, #2 presubscript, #3 tensor symbol, #4 postsuperscript, #5 postsubscript
    \sideset{^{#1}_{#2}}{^{#4}_{#5}}{\mathop{#3}}
}
\usepackage{chem_braket}
% Appendix 2
\usepackage{subcaption}
\usepackage[ruled, vlined, linesnumbered, commentsnumbered]{algorithm2e}
\usepackage{tikz}
\newcommand*\circled[1]{\tikz[baseline=(char.base)]{
  \node[shape=circle,draw,fill=black,text=white,font=\bf,inner sep=0.5pt] (char)
  {\scriptsize#1};
}}

\newcommand*\circledwhite[1]{\tikz[baseline=(char.base)]{
  \node[shape=circle,draw,fill=white,text=black,font=\bf,inner sep=0.5pt] (char)
  {\scriptsize#1};
}}
\usepackage{makecell}
%%%%%%%%%%%%%%%%%%%%%%%%%%%%%%%%%%%%%%%%%%%%%%%%%%%%%%%%%%%%%%%%%%%%%%%%%%%%%%%%%%%%%%%%%%%%%%%%%%%
% Change Figure and Table Numeration
%%%%%%%%%%%%%%%%%%%%%%%%%%%%%%%%%%%%%%%%%%%%%%%%%%%%%%%%%%%%%%%%%%%%%%%%%%%%%%%%%%%%%%%%%%%%%%%%%%%
%\chapterfloats %Uncomment this to get figures and tables numbered within chapters.

%%%%%%%%%%%%%%%%%%%%%%%%%%%%%%%%%%%%%%%%%%%%%%%%%%%%%%%%%%%%%%%%%%%%%%%%%%%%%%%%%%%%%%%%%%%%%%%%%%%
\title[Theoretical Treatment of Weakly Bound Fermions to Atoms, Molecules, and Clusters]{Theoretical Treatment of Weakly Bound Fermions to Atoms, Molecules, and Clusters}
\author{Shiv Upadhyay}
\degree{Master of Science, Duquesne University, 2017\\Bachelor's of Arts, Washington and Jefferson College, 2015}
\school{Dietrich School of Arts and Sciences}
\date{April 20, 2023}
%\year{2022}
\setyear{2023}
\keywords{hail-to-pitt, pittetd, theses, format}
\subject{Dissertation}
%%%%%%%%%%%%%%%%%%%%%%%%%%%%%%%%%%%%%%%%%%%%%%%%%%%%%%%%%%%%%%%%%%%%%%%%%%%%%%%%%%%%%%%%%%%%%%%%%%%
\committeemember{Kenneth D. Jordan, Richard King Mellon Professor and Distinguished Professor of
Computational Chemistry, Department of Chemistry}
\committeemember{Geoffrey Hutchison,  Associate Professor, Department of Chemistry}
\committeemember{Peng Liu, Associate Professor, Department of Chemistry}
\committeemember{David Yaron, Professor, Department of Chemistry, Carnegie Mellon University}
%%%%%%%%%%%%%%%%%%%%%%%%%%%%%%%%%%%%%%%%%%%%%%%%%%%%%%%%%%%%%%%%%%%%%%%%%%%%%%%%%%%%%%%%%%%%%%%%%%%
\raggedbottom

