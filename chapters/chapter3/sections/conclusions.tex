\section{Conclusions}
% TODO edit here
In this study we have applied various EOM-CC methods and two different quantum Monte Carlo methods to calculate the EBE of a model \ce{(H2O)4} cluster at two geometries, one at which the anion is bound in the HF approximation and the other at which it is not.
Diffusion Monte Carlo calculations using single determinant trial functions based on Hartree-Fock orbitals are shown to bind the excess electron even when the initial wave function for the anion has collapsed onto the neutral plus discretized continuum orbital. 
However, such calculations significantly underestimate the EBE, whereas SD DMC calculations using trial wave functions for the anion with a more realistic charge distribution for the excess electron give larger EBE values that are in close agreement with our best estimate EOM-CCSDT values for both geometries considered.

For R = \SI{4}{\angstrom}, at which the anion is correlation bound, use of such trial wave functions accurately reflecting the physical charge density resulted in AFQMC-predicted EBE values in agreement with the EOM-CCSD(T)(a)$^*$ result (when using comparable basis sets). 
However, at R = \SI{7}{\angstrom}, AFQMC calculations with HF trial wave functions significantly overestimate the EBE compared to EOM-CC and DMC values, suggesting the need for an improved trial wave functions in this case.
For the \ce{(H2O)4} model system, the restricted SDCI represents an economical way to create trial wave functions for QMC calculations on non-valence anions that are not bound in the Hartree-Fock approximation. 
However, it remains to be seen if this strategy will be as effective for systems in which the neutral species is more strongly correlated than the model \ce{(H2O)4} cluster.

Finally, we note that at R = \SI{4}{\angstrom}, for which the anion is NVCB in nature, the most frequently used method to characterize such anions, EOM-CCSD, underestimates the EBE by about 10\% compared to the result of EOM-CCSDT calculations. 
Both DMC and AFQMC are viable alternatives to high order EOM methods, and while more computationally demanding for the \ce{(H2O)4} cluster, they demonstrate lower scaling with system size than EOM methods, making them attractive for the characterization of non-valence anions of much larger systems.

