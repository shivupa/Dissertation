\section{Summary}
% TODO edit here
The diffusion Monte Carlo (DMC), auxiliary field quantum Monte Carlo (AFQMC), and equation-of-motion coupled cluster (EOM-CC) methods are used to calculate the electron binding energy (EBE) of the non-valence anion state of a model \ce{(H2O)4} cluster.
 Two geometries are considered, one at which the anion is unbound and the other at which it is bound in the Hartree-Fock (HF) approximation.
 It is demonstrated that DMC calculations can recover from the use of a HF trial wave function that has collapsed onto a discretized continuum solution, although larger electron binding energies are obtained when using a trial wave function for the anion that provides a more realistic description of the charge distribution, and, hence, of the nodal surface.
 For the geometry at which the cluster has a non-valence correlation-bound anion, both the inclusion of triples in the EOM-CC method and the inclusion of supplemental diffuse d functions in the basis set are important.
 DMC calculations with suitable trial wave functions give EBE values in good agreement with our best estimate EOM-CC result.
 AFQMC using a trial wave function for the anion with a realistic electron density gives a value of the EBE nearly identical to the EOM-CC result when using the same basis set.
 For the geometry at which the anion is bound in the HF approximation, the inclusion of triple excitations in the EOM-CC calculations is much less important.
 The best estimate EOM-CC EBE value is in good agreement with the results of DMC calculations with appropriate trial wave functions.

