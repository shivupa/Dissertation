\section{Summary}
% TODO edit here
Neutral molecules with sufficiently large dipole moments can bind electrons in diffuse nonvalence orbitals with most of their charge density far from the nuclei, forming so-called dipole-bound anions. 
Because long-range correlation effects play an important role in the binding of an excess electron and overall binding energies are often only of the order of 10-100s of wave numbers, predictively modeling dipole-bound anions remains a challenge.
Here, we demonstrate that quantum Monte Carlo methods can accurately characterize molecular dipole-bound anions with near threshold dipole moments.
We also show that correlated sampling Auxiliary Field Quantum Monte Carlo is particularly well-suited for resolving the fine energy differences between the neutral and anionic species.
These results shed light on the fundamental limitations of quantum Monte Carlo methods and pave the way toward using them for the study of weakly-bound species that are too large to model using traditional electron structure methods. 
