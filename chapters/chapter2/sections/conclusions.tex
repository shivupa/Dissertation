\section{Conclusions}
In this work, we have demonstrated that low-scaling \gls{qmc} methods, and in particular, C-\gls{afqmc}, are capable of resolving the fine energy differences required to accurately predict electron binding energies of dipole-bound anions.
Electron binding energies within wave numbers of previous coupled cluster and experimental results were obtained for HCN, CH$_{2}$CHCN, CH$_{3}$CN, C$_{3}$H$_{2}$, and C$_{3}$H$_{2}$O$_{3}$.
Our results demonstrate that, while uncorrelated \gls{dmc} and \gls{afqmc} methods can \textit{qualitatively} describe dipole-bound species, only correlated sampling techniques such as C-\gls{afqmc} are capable of achieving \textit{quantitative} accuracy within computationally tractable amounts of time.
The success of C-\gls{afqmc} in this work beckons for the further development of correlated \gls{dmc} methods capable of resolving the molecular energy differences that lie at the heart of all chemical processes.
These findings pave the way toward using stochastic methods to study the much larger polycyclic aromatic hydrocarbons (PAH) \cite{10.1021/acs.jpclett.5b01858} and long-chain carbon anions\cite{10.1086/311637} thought to contribute to the diffuse interstellar bands, as well as correlation-bound anions\cite{10.1021/jz400195s,10.1063/1.4991497} and weakly bound clusters \cite{10.1073/pnas.1715434115} whose size puts them beyond reach of most high accuracy methods.
