\section{Computational Methods}
% TODO edit here
In order to compute the EBEs, the neutral geometries were first optimized using the MP2 method\cite{10.1103/PhysRev.46.618} together with the aug-cc-pVDZ basis set\cite{10.1063/1.456153,10.1063/1.462569} in Gaussian 09.\cite{10.1016/0009-26149080029-D, 10.1016/0009-26149080030-H,10.1016/0009-26148885250-3, 10.1016/0009-26148987442-1,10.1016/0009-26149400116-2}
Hartree-Fock (HF) wave functions were then generated in Gaussian 09, GAMESS,\cite{10.1002/jcc.540141112,10.1016/B978-044451719-7/50084-6} or NWChem\cite{10.1016/j.cpc.2010.04.018} for use as trial wave functions, which guide sampling and curb the growth of the sign/phase problem, an exponential decay in the signal to noise ratio, in DMC and AFQMC \cite{10.1063/1.431514,10.1007/BF01030009}.
In order to obtain stable dipole-bound anions, it is essential to use flexible basis sets with very diffuse basis functions. 
Here, we use the aug-cc-pVDZ basis set augmented with a set of diffuse $s$ and $p$ functions, and in one case, also $d$ functions, located near the positive end of the dipole\cite{10.1103/PhysRevA.54.1906}. 
DMC\cite{10.1103/RevModPhys.73.33, 10.1088/0953-8984/22/2/023201, 10.1016/bs.aiq.2015.07.003} and AFQMC\cite{10.1002/wcms.1364, Zhang_AFQMC_BookChapter} are performed on all of the species studied. %no doi for this citation
DMC calculations were conducted using the CASINO package\cite{10.1088/0953-8984/22/2/023201,10.1103/PhysRevB.72.085124}. Averages were obtained by sampling the configurations of 4650 walkers for 500,000 or more propagation steps. 
In order to obtain DMC energies in the zero time-step ($\Delta \tau \rightarrow 0$) limit, DMC simulations were conducted at three different time step sizes and then linearly extrapolated to zero time-step to yield the final values reported. 

Because the calculation of dipole-bound anion vertical binding energies involves energy differences between two species with identical molecular geometries, we employed C-AFQMC \cite{10.1021/acs.jctc.7b00224} for the majority of our AFQMC calculations. In this approach, differences between two quantities normally computed separately using independently generated auxiliary field configurations are instead computed using the same set of configurations. For sufficiently similar systems like many dipole-bound anions and their parent neutral molecules, this can result in a systematic cancellation of errors, which markedly reduces the variance associated with calculated observables. In our calculations, a set of randomly-seeded repeat simulations were initialized, and after an initial equilibration period, the mean and standard error of the cumulative averages were computed among the repeats.  Convergence is attained when the mean is visually observed to plateau, and when the statistical error falls below a target threshold.  The $\Delta\tau \rightarrow 0$ limit was estimated via linear extrapolations using simulations performed at $\Delta \tau=0.01$ and $\Delta \tau=0.005$ a.u. 


