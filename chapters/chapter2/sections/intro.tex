\section{Introduction}
% TODO edit here
Dipole-bound anions are intriguing species that bind excess electrons via their molecular dipole moments \cite{Simons_JPCA_2008,Jordan_Wang_Ann_Rev_2003}. As the charge-dipole attraction is governed by a long-range potential that behaves as $1/r^{2}$ at large $r$, dipole-bound electrons are delicately bound in diffuse orbitals with most of their charge density located far from the atomic centers of their parent molecules\cite{Jordan_Luken_JCP_1976,Gutowkshi_Jordan_PRA_1996,Barnett_JCP_1988}. Within the Born-Oppenheimer approximation, the critical dipole moment necessary for binding an electron is 1.625 D, \cite{Fermi_Teller_PhysRev_1947,Turner_Anderson_PhysRev_1968,Crawford_ProcPhysSoc_1967} but increases to 2.5 D or larger when corrections to the Born-Oppenheimer approximation are made\cite{Garrett_CPL_1970,Garrett_PRA_1971,Lykke_PRL_1984,Desfrancois_Schermann_PRL_1994}. Beyond being ``doorways'' to the formation of valence-bound anions\cite{Hendricks_JCP_1998,Compton_JCP_1996,Desfrancois_JCP_1996,Desfrancois_JPCA_1998}, dipole-bound anions may be key contributors to the diffuse interstellar bands, a set of absorption peaks emanating from the interstellar medium whose source has yet to be conclusively identified \cite{Sarre_MNRAS_2000, Sarre_JMS_2006, Guthe_AstroPhysJ_2001, Maier_Nagy_APJ_2011, McCall_AstroPhysJ_2002, Fortenberry_JCP_2011, Larsson_RepProgPhys_2012}. The sheer experimental challenge of resolving the exceedingly small binding energies of such fragile species has motivated spectroscopists to produce dipole-bound species via electron attachment\cite{Hendricks_JCP_1996,Buytendyk_JCP_2016} and Rydberg electron transfer,\cite{Desfrancois_Schermann_PRL_1994,Desfrancois_IntJModPhysB_1996,Hammer_JCP_2003} and to study them via field detachment and photoelectron spectroscopy \cite{Wang_RevSciInstr_1999}. From the theoretical perspective, dipole-bound anions are of particular interest because they pose a formidable challenge for \textit{ab initio} methods -- only high levels of theory, such as coupled cluster theories combined with large, flexible basis sets, are capable of accurately predicting dipole-bound anion electron binding energies that are often of the order of just a few hundred wave numbers in magnitude\cite{Gutowski_PRA_1996,Jordan_Wang_Ann_Rev_2003,Gutowski_Simons_IJQC_1997,Nooijen_JCP_1995,Gutowkshi_Jordan_PRA_1996,Gutzev_CPL_1997}. However, these highly accurate methods scale steeply with system size, severely restricting the size of systems to which they can be applied.  

Herein, we explore the accuracy with which Diffusion Monte Carlo (DMC)\cite{Foulkes_Rajagopal_RMP_2001,Morales_JCTC_2012,Petruzielo_JCP_2012} and Auxiliary Field Quantum Monte Carlo (AFQMC),\cite{Motta_Zhang_2018,Suewattana_Zhang_PRB_2007,Purwanto_JCP_2015,AlSaidi_PRB_2006,AlSaidi_JCP_2006_2,AlSaidi_JCP_2007} two highly accurate, stochastic methods that scale as only $O(N^{3})-O(N^{4})$ with system size, can model dipole-bound anions, with the aim of uncovering a new set of approaches for modeling dipole- and correlation-bound\cite{Voora_JCP_2017} anions of large molecules. Interestingly, despite the different approximations they employ, we find that both methods reproduce experimental results and distinguish molecules that bind an extra electron from those that do not. Furthermore, we find that a newly-developed correlated sampling AFQMC approach (C-AFQMC) \cite{Shee_Reichman_JCTC_2017} is particularly well-suited for the task of studying energy differences involving weakly-bound species and converges electron binding energies orders of magnitude faster than stochastic methods that do not employ such sampling.

To gauge the viability of characterizing dipole-bound anions using QMC methods, we compute vertical electron affinities for several systems known to form dipole-bound anions. Vertical electron affinities may be obtained by taking the difference between the energies of the neutral and anionic species both calculated at the neutral geometry. However, since the dipole-bound excess electron makes almost no impact on the geometry of the molecule, we can equate these electron affinities to the electron binding energies (EBEs), which are typically defined using the geometry of the anion. In the following paragraphs, we summarize the calculations performed; further details may be found in the Supplemental Information. 
