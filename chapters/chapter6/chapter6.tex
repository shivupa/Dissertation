\chapter{Conclusions}

In this work, the importance of correlation effects in the binding of Fermions was explored.
The first part of this work involved the study of \ce{e-}-\ce{e-} correlation effects in the diffuse anionic states of nonvalence anions.
First, the binding of an electron in an dipole field was studied using quantum Monte Carlo methods.
It was found that quantum Monte Carlo methods are viable theoretical methods for the description of \glsxtrlongpl{dba}.
This motivated the study of a model \ce{(H2O)4} nonvalence correlation bound anion system.
This model system exhibits a crossover between a correlation-bound anion and electrostatically-bound anion by tuning a single geometric parameter.
In this model system, \gls{qmc} methods were shown to provide quantitavely accurate electron binding energies in both the electrostatically-bound region and the correlation-bound region.
In this work, a compact correlated ansatz was developed that yielded excellent nodal surfaces for \gls{qmc} calculations.
This work was then extended to positron binding to atoms and molecules.
In a similar manner to the nonvalence correlation bound anion states, we developed a physically motivated CI ansatz for trial wavefunctions.
This improved the nodal surfaces for subsequent DMC calculations.
In the future, we envision that QMC will become the method of choice for the description of positron bound states.
The favorable scaling with respect to the number of electrons and the ability to capture electron-positron correlation accurate will motivate this choice.
The lessons learned and tools developed in the current work will serve as an important step in this direction.
The chemisorption of a hydrogen atom on a graphene sheet was studied.
This is a relatively weak interaction, and so the accurate description of electron correlation was crucial in this study.
