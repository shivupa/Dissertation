\chapter{Conclusions}

In this work, the importance of correlation effects in the binding of Fermions was explored.
The first part of this work involved the study of \ce{e-}/\ce{e-} correlation effects in the diffuse anionic states of nonvalence anions.
First, the binding of an electron in an dipole field was studied using quantum Monte Carlo methods.
It was found that quantum Monte Carlo methods are viable theoretical methods for the description of nonvalence anions.
This motivated the study of a more difficult nonvalence correlation bound anion system.
To this end, a model \ce{(H2O)4} cluster was studied using \gls{qmc}.
In this model system, \gls{qmc} provided quantitative electron binding energies in both the electrostatically bound region and the correlation bound region.
In this work, a compact correlated ansatz was developed that yielded excellent nodal surfaces for \gls{qmc} calculations.
In a slight detour from Fermion bound states, the chemisorption of a hydrogen atom on a graphene sheet was studied next.
The chemisorption of hydrogen is a relatively weak interaction, and so the accurate description of electron correlation was crucial in this study. 
Finally, the formation of positron bound states was studied.
In a similar manner to the nonvalence correlation bound anion states, we developed a physically motivated \gls{ci} ansatz.
This improved the nodal surfaces for subsequent \gls{qmc} calculations.

In the future, we envision that \gls{qmc} will become the method of choice for the description of positron bound states.
The favorable scaling with respect to the number of electrons and the ability to capture electron-positron correlation accurate will motivate this choice.
The lessons learned and tools developed in the current work will serve as an important step in this direction.
