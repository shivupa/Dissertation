\section{Introduction}
Positron chemistry can yield information about the electronic and vibrational properties of molecules.\citehere
Theoretical methods can yield insight into the nature of the interactions between matter and antimatter.
Specifically, the properties of interest are the energy of a positron-matter bound state and the \ce{e+}/\ce{e-} annihilation rate.
In this work, we will focus on the formation and energetics of the bound state.

Multicomponent methods are a powerful tool to describe matter-antimatter interactions.
They are also conceptually similar to standard electronic structure methods.
The origins of multicomponent methods can be traced to Thomas for the description of quantum nuclei,\cite{10.1103/PhysRev.185.90, 10.1016/0009-26146987015-6, 10.1103/PhysRevA.2.1200,10.1103/PhysRevA.3.565} and since then nuclear quantum effects have driven much of the development and application of multicomponent methods.\cite{10.1080/00268977400102681, 10.1103/PhysRevA.16.640, 10.1016/0009-26148680493-6,10.1103/PhysRevA.36.1544,10.1063/1.461538,10.1063/1.462259,10.1063/1.463827,10.1021/cr00022a003,10.1002/qua.560550305,10.1103/PhysRevLett.83.2541,10.1063/1.1288376,10.1063/1.1342757,10.1103/PhysRevLett.88.033002,10.1063/1.1457435,10.1103/PhysRevLett.89.073001,10.1039/B211193D,10.1063/1.1537719,10.1063/1.1786580,10.1063/1.1884602,10.1063/1.1891707,10.1063/1.2012332,10.1063/1.2047487,10.1063/1.2209691,10.1063/1.2244563,10.1063/1.2236113,10.1063/1.2735305,10.1063/1.2736699,10.1063/1.2755767,10.1103/PhysRevA.76.052506, 10.1103/PhysRevA.77.022506, 10.1063/1.2834926, 10.1002/SICI1097-461X199869:5<629::AID-QUA1>3.0.CO;2-X, 10.1002/SICI1097-461X199870:4/5<659::AID-QUA12>3.0.CO;2-Y,10.1063/1.479921, 10.1016/S0009-26149800519-3,10.1080/00268979909483065,10.1002/qua.21584, 10.1002/qua.21584,10.1103/PhysRevLett.101.153001,10.1063/1.2943144,10.1021/jp7098015,10.1016/S0009-26140101286-6,10.1016/S0009-26140200881-3,10.1080/713715956,10.1080/713715956,10.1016/S0166-12800300147-7,10.1016/S0166-12800300147-7,10.1063/1.1805493,10.1016/j.cplett.2004.03.091,10.1016/j.cplett.2004.03.091,10.1143/JPSJ.74.3112,10.1016/j.chemphys.2005.03.007,10.1039/B500620A,10.1063/1.2151897, 10.1021/jp0615656,10.1021/jp0615656,10.1063/1.2352753,10.1063/1.2352753,10.1063/1.2403857,10.1088/0953-8984/19/36/365235,10.1002/qua.21540, 10.1016/j.chemphys.2008.10.027, 10.1063/1.2917149,10.1063/1.3028540,10.1063/1.3028540,10.1002/qua.1106,10.1063/1.1528951,10.1063/1.1871914,10.1016/j.cplett.2006.01.064,10.1063/1.2193513,10.1021/ct6002065,10.1002/qua.21430,10.1080/00268970701618416,10.1002/jcc.20840,10.1063/1.1494980,10.1063/1.1569913,10.1103/PhysRevLett.92.103002,10.1016/j.chemphys.2004.06.009,10.1016/j.cplett.2005.01.115,10.1063/1.1940634,10.1063/1.1990116,10.1063/1.2039727,10.1021/jp053552i,10.1021/jp0634297,10.1021/jp057014h,10.1021/jp065569m,10.1021/jp0682661,10.1021/jp0704463, 10.1063/1.3236844, 10.1021/ct200473r, 10.1063/1.4709609,10.1063/1.4996038,10.1021/acs.jpclett.7b01442,10.1021/acs.jpclett.7b01442,10.1063/1.5037945,10.1021/acs.jpclett.8b00547,10.1063/1.5119124,10.1063/1.5099093, 10.1021/acs.jctc.8b01120,10.1063/1.5094035,10.1063/1.4921303,10.1063/1.4921304,10.1063/1.4812257, 10.1021/acs.jctc.2c00701, 10.1063/5.0071423, 10.1063/5.0076006, 10.1063/5.0006743, 10.1021/acs.jctc.9b01273,10.1021/acs.jctc.0c01191,10.1021/acsomega.2c07782,10.1021/acs.jpclett.0c00090, 10.1021/acs.jpclett.9b01803, 10.1021/jp810410y, 10.1039/C7CP04936F, 10.1063/1.4984098}
Soon after the first applications of multicomponent methods for nuclear quantum effects, a similar multicomponent method was used to treat positron bound states.\cite{10.1088/0022-3700/11/16/001, 10.1088/0022-3700/12/15/007,10.1063/1.438933,10.1063/1.442211,10.1088/0022-3700/14/22/019}
Since then many positronic systems have been studied using multicomponent methods of varying accuracy and computational costs.\cite{10.1103/PhysRevA.48.1903,10.1021/jp9528166, 10.1002/SICI1097-461X199870:3<491::AID-QUA5>3.0.CO;2-P, 10.1016/S0039-60289900551-8,10.1021/jp7098015,10.1016/S0009-26140101286-6, 10.1080/00268970110099602, 10.1016/S0009-26140300414-7,10.1021/jp065759x,10.1063/1.5116113, 10.1016/j.cplett.2012.04.062,10.1021/acs.jpca.6b10124,10.1002/anie.201800914,10.1063/1.4895043, 10.1103/PhysRevA.89.052709, 10.1039/C9SC04433G, 10.1021/acs.jctc.1c01193, 10.1039/D2SC04630J, 10.1021/acs.jpcb.1c10124, 10.1088/1742-6596/635/3/032119}

The binding of a positron occurs in a diffuse nonvalence orbital, and are antimatter analogues of nonvalence anions.\citehere
In previous work on nonvalence anions, we have demonstrated the importance of \ce{e-}/\ce{e-} correlation effects for such states.\cite{10.1063/5.0030942}
Positron bound states are even more sensitive to \ce{e-}/\ce{e+} correlation effects as they have a favorable coulombic interaction and the lack of an exchange interaction between the positron and the electrons.

In this work, we aim to capture \ce{e-}/\ce{e+} correlation effects by developing accurate multicomponent wave function methods and then use these accurate correlated wave functions as trial wave functions for subsequent quantum Monte Carlo (QMC) calculations.
A critical contribution of our approach is demonstrating the viability of a physically motivated configuration interaction ansatz, which incorporates the \ce{e-}/\ce{e+} dispersion interaction.
This is motivated by a similar ansatz constructed for nonvalence correlation bound (NVCB) anions.\cite{10.1063/5.0030942}

