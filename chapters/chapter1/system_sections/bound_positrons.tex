\subsection{Positron bound states}
Since the theoretical prediction\cite{10.1098/rspa.1928.0023} and subsequent experimental observation\cite{10.1126/science.76.1967.238} of positrons, the exotic antimatter has seen use in a variety of fields such as positron emission tomography in the medical field or in defect characterization in condensed matter physics.\cite{10.1088/0031-9155/51/13/R08,10.1146/annurev-bioeng-071114-040723,10.1146/annurev.ms.10.080180.002141,10.1016/j.matchar.2021.110952}
Modern applications have also included the formation of antihydrogen as a test of fundamental physics regarding violations of charge, parity, and time-reversal symmetries.\cite{10.1126/science.aaf6702,10.1038/nature24048,10.1103/PhysRevLett.110.140406,10.1038/nature14861,10.1103/PhysRevLett.82.3198}

In chemistry, one is interested in the behavior of matter.
It may seem that the study of antimatter is of little interest to chemistry, however through electron-positron interactions serve as an experimental probe to understand the electronic and vibronic structure of molecules.

In order to understand how positrons yield insight into the behavior of molecules, it helps to consider the interaction of a low energy positron with an idealized molecule.
The simplest picture of this process would imply that the positron and electron annihilation rate would be proportional to the number of electrons present.\cite{10.1103/RevModPhys.82.2557}
Interestingly, the observed rate of annihilation is orders of magnitude greater for most molecules.\cite{10.1103/RevModPhys.82.2557, 10.1103/PhysRev.140.A8, 10.1103/PhysRev.138.B216, 10.1088/0953-4075/39/17/L03, 10.1103/PhysRevLett.14.935, 10.1103/PhysRevA.77.060702, 10.1103/PhysRevLett.99.133201}
The major causes of this increased annihilation rate is the formation of positron-molecule bound states and positron-molecule temporary states (vibrational Feschbach resonances).\cite{10.1103/RevModPhys.82.2557, 10.1103/PhysRevLett.99.133201, 10.1103/PhysRevLett.88.043201,10.1103/PhysRevA.67.032706}
Since the process is molecule dependent, the experimental characterization of positron annihilation allows one to characterize different molecules.
This also explains the motivation to computationally study positron bound states.
The diffuse positronic states are diffuse analogues to nonvalence anions with direct connection to experiment.
