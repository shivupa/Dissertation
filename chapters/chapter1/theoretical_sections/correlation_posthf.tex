\subsubsection{Wave Function Based Methods}
Wave function based correlated methods build upon a reference, usually the uncorrelated \gls{hf} referenc,e and attempt to reincorporate electron correlation.
In \gls{hf}, one assumes the wave function can be written as a single \glsxtrlong{sd}.
This signel electronic configuration misses electron correlation which can be divided into two categories:
\begin{itemize}
\item Static correlation - Some wave functions cannot be written as a single configuration. If there are other electronic configurations of the same symmetry with similar energy, then the true wave function for the system will have a nonneglible contribution from the low lying electronic configurations. A well-known example is the \ce{Be} atom. The $1s^2 2s^2$ configuration is not significantly lower in energy that the $1s^2 2p^2$ electronic configuration, and therefore these configurations contribute to the true wave function.\cite{10.1063/1.1669638, 10.1103/PhysRevA.4.908, 10.1103/PhysRevA.14.1965, 10.1103/PhysRevA.65.042507, 10.1063/1.477970}

\item Dynamic correlation - Since an averaged interaction potential is used in \gls{hf}, the instataneous interactions between electrons are neglected. In a time indepenedent picture this is manifested by a poorly resolved electron-electron cusp. Exact conditions exist for the functional form of the wave function as two electrons collaesce, but \gls{hf} doesn't obey these Kato cusp conditions.\cite{10.1002/cpa.3160100201} Including additional electronic configurations can improve the description of this cusp.
\end{itemize}

When a nonminimal basis, meaning there are more basis functions than electrons, is used in \gls{hf}, the resulting wave function has occupied orbitals and unoccupied virtual orbitals.
Other determinants can be constructed from the \gls{hf} determinant by ``exciting'' an electron from an occupied to unoccupied orbital.
Post-\gls{hf} methods write an ansatz for the correlated wave function to include these excitations.

\subsubsection{Full Configuration Interaction}
Conceptually the most simple approach is to construct all of the possible excitations from the \gls{hf} determinant and to write the wave function as a superposition of all of these determinants,
\begin{equation}
\ket{\Psi} = C_0 \ket{\Phi_0} + \sum C \ket{\Phi_{\mu}^{a}} + \sum C \ket{\Phi_{\mu\nu}^{ab}} + \cdots.
\end{equation}
With all possible excitations are generated, the resulting wave function is called the \gls{fci} wave function.
Once the ansatz has been contructed the coefficients are variationally optimized yielding the correlated wave function.
Unfortunately the complexity of the \gls{fci} wave function means this method scales as $N!$ where $N$ is the number of electrons.
This restricts its usage to small systems with a small number of electrons and orbitals.

\subsubsection{Truncated Configuration Interaction}
The FCI wave function may also be written using excitation operators that generate the excited electronic determinants,
\begin{equation}
    \ket{\Psi} = (C_0 \hat{1} + \sum C \hat{R}_{\mu}^{a} + \sum C \hat{R}_{\mu\nu}^{ab} + \cdots )\ket{\Phi_0},
    \label{eq:ci_tci}
\end{equation}
where $\hat{R}$ are the excitation operators.
Truncated \gls{ci} expansions work by limiting the sum of excitation operators.
Truncated \gls{ci} expansions are commonly created by terminanting the sum in eq.~\ref{eq:ci_tci} at a certain excitation level.
This forms a family of truncated \gls{ci} expansions such as \gls{cis}, \gls{cisd}, \gls{cisdt}, and higher levels.

Although this the most common truncated \gls{ci} expansion, other truncations exist.
For example, senority \gls{ci} truncates based on the number of unpaired electrons.
The most common variant of this type of truncated \gls{ci} is \gls{doci}, which is a senority zero \gls{ci} meaning there are no unpaired electrons.\cite{10.1021/j100818a001, 10.1063/1.1701519, 10.1063/1.1841109, 10.1021/jp963953l,10.1021/jp963953l}
\gls{doci} is not a widely used method compared to excitation based truncated \gls{ci}, but senority based methods have been experiencing renewed interest.\cite{10.1021/jp963953l,10.1063/1.5130660,10.1063/1.4904384,10.1021/ct300902c,10.1080/00268976.2013.874600,10.1021/jp502127v,10.1021/jp502127v,10.1063/1.4880819,10.1021/acs.jpclett.2c00730}

The difficulty with truncated \gls{ci} schemes is that one doesn't have an \textit{a priori} knowledge of the relevant determinants.
Selected \gls{ci} methods, such as \gls{cipsi},\cite{10.1063/1.1679199} attempt to address this deficiency by growing the \gls{ci} space iteratively.
In the first iteration, single and double excitations from the \gls{hf} determinant are considered.
The excitations are given a perturbative estimate of their importance, and if the estimated importance is greater than a threshold the excitation is retained.
The \gls{ci} eigenvalue is solved in the space of the retained determinants to obtain variational values of each determinant's contribution.
The cycle is then restarted where single and double excitations are generated from the previous iterations retained determinants.
The iterative process continues until the \gls{ci} energy drops below a predefined threshold.
This cyclic growth allows for the generation of compact \gls{ci} expansions with mainly important determinants selected.

\subsubsection{Coupled cluster methods}
\Gls{cc} methods eschew the linear ansatz of \gls{ci}, and instead adopt an exponential ansatz,
\begin{equation}
    \ket{\Psi} = e^{\hat{T}_1 + \hat{T}_2 + \cdots}\ket{\Phi_0}.
    \label{eq:cc_ansatz}
\end{equation}
This exponential ansatz has the advantage of being size consistent unlike \gls{ci} methods.
Size consistency is a property of quantum mechanical methods where the energy of two noninteracting systems  is equal to the sum of each subsystem's energy,
\begin{equation}
    E(A+B) = E(A) + E(B).
\end{equation}
Truncated \gls{ci} would correlate the subsystems more and yield lower energies for the subsystems.
\gls{cc} by the nature of the exponential ansatz has a multiplicatively separable energy, which leads to size consistency.\cite{10.1002/wcms.76,10.1146/annurev.pc.32.100181.002043}

The extension of coupled cluster to excited states is slight more complicated than the \gls{ci} case. 
For \gls{cc} excited states, the \gls{eom} formulation has be developed, which conceptually is similar to performing a \gls{ci} calculation on top of a \gls{cc} wave function.\cite{10.1016/0009-26149389023-B, 10.1063/1.464746}
This ansatz is similar to the \gls{ci} ansatz with excitations applied to the \gls{cc} wave function,
\begin{equation}
    \ket{\Psi} = (C_0 \hat{1} + \sum C \hat{R}_{\mu}^{a} + \sum C \hat{R}_{\mu\nu}^{ab} + \cdots ) e^{\hat{T}_1 + \hat{T}_2 + \cdots}\ket{\Phi_0}.
\end{equation}
In practice, this can be implemented similar to a configuration interaction calculation where the matrix elements come from a similarity transformed Hamiltonian.
This method is especially powerful as the excitation operators do not need to conserve particle number so one can calculate ionization potentials and electron affinities as well.\cite{10.1063/1.468592, 10.1016/0009-26148987149-0, 10.1063/1.468022, 10.1063/1.469817}
