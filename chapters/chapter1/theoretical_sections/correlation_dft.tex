\subsubsection{Density Functional Theory Methods}
\gls{dft} has seen widespread use since it is an economical, albeit approximate, treatment of electron correlation.\cite{10.1103/RevModPhys.87.897, 10.1002/qua.24259}

\gls{dft} in theory is an exact method.
The first Hohenberg-Kohn theorem establishes an exact mapping between the interaction potential and the electron density.\cite{10.1103/PhysRev.136.B864}
The second Hohenberg-Kohn theorem shows that a variational principle for the energy as a functional of the density exists.\cite{10.1103/PhysRev.136.B864}
While these theorems and extensions for finite temperature, magnetic fields, and other generalizations,\cite{10.1103/PhysRev.137.A1441, 10.1103/PhysRevLett.59.2360,10.1073/pnas.76.12.6062} solidify \gls{dft}'s theoretical footing, they do not indicate how one would practically use \gls{dft}.

In practice, the Kohn-Sham equations are used which maps the density-based Schr{\"o}dinger equation problem on to a fictious noninteracting problem with an effective field,
\begin{equation}
(\hat{T} + \hat{V}_{\mathrm{eff}}) \psi(r) = E \psi(r).
\end{equation}
This formulation resembles a \gls{hf} equation where the interaction potential has been replaced by an effective potential.
This allows one to utilize the standard \gls{hf} machinery and incorporate electron correlation at minimal cost.
The form of the effective potential is what is commonly referred to as the \gls{dft} functional.
These functionals are constructed and parameterized as the exact functional is not known.
This means that although \gls{dft} is an exact theory, in practice it is not \textit{ab initio}.


