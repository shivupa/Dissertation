\subsection{Mean Field Methods}
The electronic time independent Schrodinger equation is still a many body problem, and due to its multidimensional nature, it does not lend itself to an easy solution.
The common approximation for many body problems in physics is a \gls{mf} approximation.
Mean-field approximations involve mapping an interacting problem onto a noninteracting problem with averaged interactions.
The \gls{hf} method involves mapping the interacting quantum many-body problem onto a noninteracting problem with each electron interacting with the average potential of the other electrons.
Since the average potential that each electron experiences is dependent on the density of each of the other electrons, the solution to the \gls{hf} equations must be found iteratively.
This means that the independent particle densities yield averaged interaction potentials that yield new independent particle densities, and this process is repeated.
Eventually, the previous iteration's densities are unchanged in an iteration, a condition termed self-consistency.
Once this condition has been fulfilled, we stop iterating, and we have found the \glsxtrlong{mf} solution.

\shiv{add hf algo}
The above described \gls{hf} algorithm for a Slater determinant trial wavefunction is given in listing 1.
The linear algebra machinery makes the \gls{hf} method amenable to solution on modern computational hardware.
