\subsection{Correlated Methods}
Once the \gls{hf} method, has converged a single Slater determinant wave function is obtained.
The description of the wave function captures many of the salient properties of the system.
For example, if we knew the exact nonrelativistic energy for a system, the \gls{hf} energy accounts for $\sim$99\% of the exact energy.
We define the missing energy as the correlation energy,
\begin{equation}
E_{\mathrm{corr}} = E_{\mathrm{exact}} - E_{\mathrm{HF}},
\end{equation}
since it is the portion of the energy missed due to the use of averaged interactions between electrons.

If the correlation energy per electron were constant, then computational chemistry would be much easier.
For example, if one were calculating the atomization energy of a water molecule,
\begin{equation}
    E_{\mathrm{atom}} = E_{\mathrm{H2O}} - 2 E_{\mathrm{H}} - E_{\mathrm{O}},
\end{equation}
each term would have a constant shift, and the relative energy difference $E_{\mathrm{atom}}$ would be equivalent if calculated with the exact energy or the \gls{hf} energy.
Unfortunately this is not the case, and thus we aim to capture the correlation energy accurately for a chemical system.

In this thesis three main approaches are used to recover electron correlation:
\begin{itemize}
    \item \textbf{Density Function Theory}- Since the \glsxtrlong{hf} method neglects $e^{-}$ correlation by using an averaged potential, density functional theory attempts to modify the effective potential to recover correlation energy.
    \item \textbf{Wave function based methods}- Build upon a \glsxtrlong{hf} wavefunction and reincorporate correlation through methods such as perturbation theory.
\item \textbf{Stochastic methods}- Randomly sample correlated forms of wave functions, which would be difficult or impossible to solve deterministically, to recover correlation.
\end{itemize}
