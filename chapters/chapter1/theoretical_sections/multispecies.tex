\subsection{Multicomponent Methods}
Multicomponent methods allow for the treatment of multiple quantum particle types in quantum chemistry.
These methods can be used to coupled the electronic degrees of freedom to the nuclear degrees of freedom by treating light nuclei, usually only hydrogen, quantum mechanically.
This allows one to study nuclear quantum effects and go beyond the \gls{boa}.
These methods can also be used to calculate the interactions between electrons and antimatter such as positrons.
The advantage of such an approach is that the multicomponent methods are conceptually similar to standard electronic structure methods.

The development of multicomponent methods has occurred over many years beginning from the seminal work of Thomas,\cite{10.1103/PhysRev.185.90, 10.1016/0009-26146987015-6, 10.1103/PhysRevA.2.1200,10.1103/PhysRevA.3.565} which was soon followed by the application of multicomponent methods to positrons.\cite{10.1088/0022-3700/11/16/001, 10.1088/0022-3700/12/15/007,10.1063/1.438933,10.1063/1.442211,10.1088/0022-3700/14/22/019}
Since then, the application and developments of multicomponent methods have continued for quantum protons,\cite{10.1080/00268977400102681, 10.1103/PhysRevA.16.640, 10.1016/0009-26148680493-6,10.1103/PhysRevA.36.1544,10.1063/1.461538,10.1063/1.462259,10.1063/1.463827,10.1021/cr00022a003,10.1002/qua.560550305,10.1103/PhysRevLett.83.2541,10.1063/1.1288376,10.1063/1.1342757,10.1103/PhysRevLett.88.033002,10.1063/1.1457435,10.1103/PhysRevLett.89.073001,10.1039/B211193D,10.1063/1.1537719,10.1063/1.1786580,10.1063/1.1884602,10.1063/1.1891707,10.1063/1.2012332,10.1063/1.2047487,10.1063/1.2209691,10.1063/1.2244563,10.1063/1.2236113,10.1063/1.2735305,10.1063/1.2736699,10.1063/1.2755767,10.1103/PhysRevA.76.052506, 10.1103/PhysRevA.77.022506, 10.1063/1.2834926, 10.1002/SICI1097-461X199869:5<629::AID-QUA1>3.0.CO;2-X, 10.1002/SICI1097-461X199870:4/5<659::AID-QUA12>3.0.CO;2-Y,10.1063/1.479921, 10.1016/S0009-26149800519-3,10.1080/00268979909483065,10.1002/qua.21584, 10.1002/qua.21584,10.1103/PhysRevLett.101.153001,10.1063/1.2943144,10.1021/jp7098015,10.1016/S0009-26140101286-6,10.1016/S0009-26140200881-3,10.1080/713715956,10.1080/713715956,10.1016/S0166-12800300147-7,10.1016/S0166-12800300147-7,10.1063/1.1805493,10.1016/j.cplett.2004.03.091,10.1016/j.cplett.2004.03.091,10.1143/JPSJ.74.3112,10.1016/j.chemphys.2005.03.007,10.1039/B500620A,10.1063/1.2151897, 10.1021/jp0615656,10.1021/jp0615656,10.1063/1.2352753,10.1063/1.2352753,10.1063/1.2403857,10.1088/0953-8984/19/36/365235,10.1002/qua.21540, 10.1016/j.chemphys.2008.10.027, 10.1063/1.2917149,10.1063/1.3028540,10.1063/1.3028540,10.1002/qua.1106,10.1063/1.1528951,10.1063/1.1871914,10.1016/j.cplett.2006.01.064,10.1063/1.2193513,10.1021/ct6002065,10.1002/qua.21430,10.1080/00268970701618416,10.1002/jcc.20840,10.1063/1.1494980,10.1063/1.1569913,10.1103/PhysRevLett.92.103002,10.1016/j.chemphys.2004.06.009,10.1016/j.cplett.2005.01.115,10.1063/1.1940634,10.1063/1.1990116,10.1063/1.2039727,10.1021/jp053552i,10.1021/jp0634297,10.1021/jp057014h,10.1021/jp065569m,10.1021/jp0682661,10.1021/jp0704463, 10.1063/1.3236844, 10.1021/ct200473r, 10.1063/1.4709609,10.1063/1.4996038,10.1021/acs.jpclett.7b01442,10.1021/acs.jpclett.7b01442,10.1063/1.5037945,10.1021/acs.jpclett.8b00547,10.1063/1.5119124,10.1063/1.5099093, 10.1021/acs.jctc.8b01120,10.1063/1.5094035,10.1063/1.4921303,10.1063/1.4921304,10.1063/1.4812257, 10.1021/acs.jctc.2c00701, 10.1063/5.0071423, 10.1063/5.0076006, 10.1063/5.0006743, 10.1021/acs.jctc.9b01273,10.1021/acs.jctc.0c01191,10.1021/acsomega.2c07782,10.1021/acs.jpclett.0c00090, 10.1021/acs.jpclett.9b01803, 10.1021/jp810410y, 10.1039/C7CP04936F, 10.1063/1.4984098}
positronic systems,\cite{10.1103/PhysRevA.48.1903,10.1021/jp9528166, 10.1002/SICI1097-461X199870:3<491::AID-QUA5>3.0.CO;2-P, 10.1016/S0039-60289900551-8,10.1021/jp7098015,10.1016/S0009-26140101286-6, 10.1080/00268970110099602, 10.1016/S0009-26140300414-7,10.1021/jp065759x,10.1063/1.5116113, 10.1016/j.cplett.2012.04.062,10.1021/acs.jpca.6b10124,10.1002/anie.201800914,10.1063/1.4895043, 10.1103/PhysRevA.89.052709, 10.1039/C9SC04433G, 10.1021/acs.jctc.1c01193, 10.1039/D2SC04630J, 10.1021/acs.jpcb.1c10124, 10.1088/1742-6596/635/3/032119}
and other quantum particle types.\cite{10.1002/qua.22069, 10.1021/acs.chemrev.9b00798,10.1021/acs.jctc.5b00879,10.1021/acs.jctc.5b00879, 10.1002/qua.24500, 10.1016/j.cplett.2012.04.062, 10.1063/1.4812259, 10.1016/j.cplett.2013.03.004, 10.1021/jp501289s, 10.1002/qua.25705}

%%%%
%  Bressanini, Dario
%  Yang Yang @ Wisconsin
%  Jun Yang @ HKU
%  Xiaosong Li
%  Marcus Reiher
%  David Sherrill
%  swann/gribakin
%%%%


