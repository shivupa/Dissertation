\subsubsection{Stochastic Methods}

\subsubsection{Projector quantum Monte Carlo}
This work involves the application of projector \gls{qmc} methods to locate ground state solutions to the imaginary time Schr{\"o}dinger equation.
The Schr{\"o}dinger equation in imaginary time ($\tau\rightarrow it$) shifted by a constant ($E_r$) is given by eq.~\ref{eq:imSE}.
\begin{equation}
\frac{\partial \ket{\Psi}}{\partial \tau} = - (\hat{H} - E_r) \ket{\Psi}
    \label{eq:imSE}
\end{equation}
The formal solution expanded in energy eigenstates is eq. \ref{eq:formsol}.
\begin{equation}
\ket{\Psi(\tau)} = \sum_{i=0}^{\infty} e^{-(\epsilon_i - E_r) \tau} \ket{\phi_i}
    \label{eq:formsol}
\end{equation}
This means any initial state with nonzero ground state overlap converges to the ground state.
The constant $E_r$ is introduced to maintain normalization. 
To illustrate this consider setting $E_r = 0$, all positive energy states would die off exponentially, but all negative energy states would grow exponentially.
The ground state would grow the fastest so in the long time limit it would dominate, but numerically this involves exponential growth.
This is mitigated by introducing the $E_r$, setting it as close to $E_0$ as possible, and periodically updating it.

The ground state obeys the projection equation (eq.~\ref{eq:proj}).
\begin{equation}
\ket{\Psi_0} \propto \lim_{\tau\rightarrow\infty} e^{-\tau (\hat{H} - E_r)} \ket{\Psi_T}
    \label{eq:proj}
\end{equation}
We now introduce a basis $\ket{B}$ with the only constraint that it is complete.
The propagation of a state in imaginary time (multiplying $\ket{B'}$ and inserting a complete basis $\int dB \ket{B}\bra{B} = 1$)
\begin{equation}
\braket{B'|\Psi(\tau)} = \int dB \braket{B'|e^{-(\hat{H}-E_r)\tau}|B}\braket{B|\Psi(0)}
    \label{eqprop}
\end{equation}
The term $\braket{B'|e^{-(\hat{H}-E_r)\tau}|B}$ is the propagator (Green's function), which is unknown. 
The short time approximation to the Green's function (eq.~\ref{eqshorttimeprop}) however can be be constructed by Suzuki-Trotter decomposition of the exponential, which is exact as $\Delta \tau \rightarrow 0$.\cite{10.1090/S0002-9939-1959-0108732-6,10.1007/BF01609348}
\begin{equation}
\begin{aligned}
    \braket{B'|\Psi(\tau)} = \int dB_{n} \cdots dB_{1} dB &\braket{B'|e^{-(\hat{H}-E_r)\Delta\tau}|B_{n}}\\
                                                          &\braket{B_{n}|e^{-(\hat{H}-E_r)\Delta\tau}|B_{n-1}}\\
                                                          &\cdots \\
                                                          &\braket{B_{2}|e^{-(\hat{H}-E_r)\Delta\tau}|B_{1}} \braket{B_{1}|\Psi(0)}
    \label{eqshorttimeprop}
\end{aligned}
\end{equation}
For \gls{dmc}, $\ket{B}$ is chosen to be real space $\ket{R}$.
For \gls{afqmc}, $\ket{B}$ is chosen to be an over-complete set of nonorthogonal determinants $\ket{D}$.
This space may seem complicated, but its use is motivated by the fact that Fermionic antisymmetry is built in.
In order to use an exponential propagator, the space must be over-complete and nonorthogonal.\cite{10.1103/PhysRevB.55.7464,10.1103/PhysRevLett.74.3652,10.1103/PhysRevLett.90.136401}
The methods differ in how the short time approximation to the Green's function is constructed.

\subsubsection{Diffusion Monte Carlo (DMC)}
\gls{dmc} performs the projection in real space.\cite{10.1016/bs.aiq.2015.07.003,10.1103/RevModPhys.73.33}
\begin{equation}
    \Psi(R',\tau + \Delta\tau) = \int_{\mathbb{R}^3} G(R \rightarrow R',\Delta \tau) \Psi(R,\tau)
\end{equation}
As previously stated the short time approximation to the Green's function can be constructed (eq.~\ref{eqshorttimedmc}).
\begin{equation}
    G(R \rightarrow R', \Delta \tau) \approx \frac{1}{(2\pi\Delta\tau)^{3N/2}} e^{-\frac{(R - R')^2 }{2\Delta\tau}} e^{-\Delta\tau[\frac{V(R) + V(R')}{2} - E_T]}
\label{eqshorttimedmc}
\end{equation}
The first exponential term represents a diffusion term and the second represents a branching process.
The stochastic evaluation of these processes is known from diffusing species undergoing chemical reaction, hence the name \glsxtrlong{dmc}.

%A problem with this approach is that the Hamiltonian used is spin independent, which means the lowest energy solution is Bosonic.
A problem with this approach is that the lowest energy solution is Bosonic, and therefore propagating with the unmodified projector (eq.~\ref{eqshorttimedmc}) leads to the Bosonic ground state.
In order to locate the Fermionic ground state, the mixed distribution $f(R,\tau) = \Psi_t(R)\Phi(R,\tau)$ is propagated where $\Psi_t$ is a trial wave function and $\Phi$ is the distribution of the ``true'' distribution.
Applying the propagator (eq.~\ref{eqprop}) to the mixed distribution results in the importance-sampled propagator (eq.~\ref{eqimpsampprop}). 
\begin{equation}
    \tilde{G}(R \rightarrow R', \Delta \tau)  = \Psi_t(R') G(R \rightarrow R', \Delta \tau) \frac{1}{\Psi_t(R)}
\label{eqimpsampprop}
\end{equation}
The importance-sampled propagator is a similarity transformation of the standard propagator.
By introducing the trial wave function $\Psi_t$, the nodes of the wave function can be fixed allowing one to locate the Fermionic ground state.
In practice, this means propagating configurations according to the short time approximation to the importance-sampled Green's function (eq.~\ref{eqimpshorttimedmc}) and rejecting any moves that cross the nodes of the trial wave function $\Psi_t$.
\begin{equation}
    \tilde{G}(R \rightarrow R', \Delta \tau) \approx \frac{1}{(2\pi\Delta\tau)^{3N/2}} e^{-\frac{(R - R' - v(R') \Delta \tau)^2 }{2\Delta\tau}} e^{-\frac{\Delta\tau}{2}[E_L(R) + E(R') - 2E_T]}
\label{eqimpshorttimedmc}
\end{equation}


%%%%%%%%%%%%%%%%%%%%%%%%%%%%%%%%%%%%%%%%%%%%%%%%%%%%%%
\subsubsection{Auxiliary Field Quantum Monte Carlo (AFQMC)}
\gls{afqmc} performs projection in Slater determinant space.
The short time approximation to the propagator consists of Suzuki-Trotter factorizations as shown in eq.~\ref{eqsuztrot}, where $H_1$ consists of the one body operators and $H_2$ is the two body electron interaction term.
\begin{equation}
e^{-\Delta\tau\hat{H}} \approx e^{-\Delta\hat{H}_1/2} e^{-\Delta\tau\hat{H}_2} e^{-\Delta\tau\hat{H}_1/2}
\label{eqsuztrot}
\end{equation}
In a determinant space, the application of a one body operator (e.g. $\hat{H}_1$) on a Slater determinant results in another Slater determinant, however in general the application of a two body operator (e.g. $\hat{H}_2$) does not.
If at each application of the propagator the number of determinants grew, then the algorithm would scale exponentially. 
To ensure that the propagation of a determinant results in a single determinant rather than a linear combination of determinants the two body operator must be linearized.
The first step of the linearization is to write the second quantized two-electron interaction operator as the square of one body terms.
\begin{equation}
    \hat{H}_2 = -\frac{1}{2} \sum_{\alpha} \hat{v}^2_{\alpha}
    \label{eqdecomp}
\end{equation}
This linearization is completed by applying the Hubbard-Stratonovich transformation to rewrite the two body operator as a collection of one body operators (eq.~\ref{eqhb}).\cite{10.1103/PhysRevLett.3.77a,zotero-4182}
\begin{equation}
e^{-\Delta\tau\hat{H}} \approx e^{-\Delta\hat{H}_0/2} \left( \int_{-\infty}^{\infty} P(\phi) \hat{B}(\phi) d\phi \right) e^{-\Delta\tau\hat{H}_0/2}
\label{eqhb}
\end{equation}
where the terms $P(\phi)$ and $\hat{B}(\phi)$ is given by
\begin{align}
    P(\phi) &= \frac{1}{2\pi} e^{-\frac{\phi^2}{2}} \\
    \hat{B}(\phi) &= e^{\sqrt{\tau} \phi \hat{v}_{\alpha}}
    \label{eqhb2}
\end{align}
This transformation maps a set of interacting particles to a set of non-interacting particles that interact with fluctuating external (auxiliary) fields.
This in practice amounts to a decomposition of the two-body electron interaction, followed by a sampling of the Gaussian distributed auxiliary fields.
Following this the short-time approximation to the propagator is constructed directly and applied to some initial wave function.

The sign problem manifests itself in a slightly different way in \gls{afqmc} than in \gls{dmc}.
The antisymmetry of the problem is fulfilled implicitly by the nature of the Slater determinant space.
However, the decomposition of the two-body operators leads to an ambiguous phase of the wave function.
To alleviate the sign problem, a trial wave function can again be introduced leading to the constrained path \gls{afqmc} method.\cite{10.1103/PhysRevB.55.7464,10.1103/PhysRevLett.74.3652}
The problems dealt with here can also be treated by invoking a special case of the constrained path AFQMC where the phase is fixed to the real plane also called phaseless \gls{afqmc}.\cite{10.1103/PhysRevLett.90.136401}


